\documentclass[a4paper, 14pt]{extarticle}
\usepackage[russian]{babel}
\usepackage[T1]{fontenc}
\usepackage{fontspec}
\usepackage{indentfirst}
\usepackage{enumitem}
\usepackage{graphicx}
\usepackage{mdframed}
\usepackage[
  left=20mm,
  right=10mm,
  top=20mm,
  bottom=20mm
]{geometry}
\usepackage{parskip}
\usepackage{titlesec}
\usepackage{xurl}
\usepackage{hyperref}
\usepackage{float}
\usepackage[
  figurename=Рисунок,
  labelsep=endash,
]{caption}

\hypersetup{
  colorlinks=true,
  linkcolor=black,
  filecolor=blue,
  urlcolor=blue,
}

\renewcommand*{\labelitemi}{---}
\setmainfont{Times New Roman}
\setmonofont{JetBrains Mono}[
  SizeFeatures={Size=11},
]

\setlength{\parskip}{6pt}

\setlength{\parindent}{1cm}
\setlist[itemize]{itemsep=0em,topsep=0em,parsep=0em,partopsep=0em,leftmargin=2.0cm,wide}
\setlist[enumerate]{itemsep=0em,topsep=0em,parsep=0em,partopsep=0em,leftmargin=2.0cm,wide}

\renewcommand{\thesection}{\arabic{section}.}
\renewcommand{\thesubsection}{\thesection\arabic{subsection}.}
\renewcommand{\thesubsubsection}{\thesubsection\arabic{subsubsection}.}

\titleformat{\section}{\normalfont\bfseries}{\thesection}{0.5em}{}
\titleformat{\subsection}{\normalfont\bfseries}{\thesubsection}{0.5em}{}

\titleformat*{\section}{\normalfont\bfseries}
\titleformat*{\subsection}{\normalfont\bfseries}

\linespread{1.5}
\renewcommand{\baselinestretch}{1.5}

\newmdenv[
  linewidth=2pt,
  topline=false,
  bottomline=false,
  rightline=false,
  skipabove=\topsep,
  skipbelow=\topsep,
]{siderules}

\begin{document}

\begin{flushright}
  \textit{Швалов Даниил, K33211}
\end{flushright}

\begin{center}
  \textbf{Ответы на экзаменационные вопросы}
\end{center}

\section*{Вопрос №1}

\begin{siderules}
  Как называются устройства, обеспечивающие преобразование сообщений в
  электрические сигналы?
\end{siderules}

Устройства, которые преобразуют сообщения в электрические сигналы, называются
цифро-аналоговыми преобразователями. Они предназначены для преобразования
данных, представленных двоичным кодом, в напряжение или ток, пропорциональные
значению цифрового кода. Существуют и обратные устройства, преобразующие
электрический сигнал в сообщения. Они называются аналого-цифровыми
преобразователями.

\section*{Вопрос №2}

\begin{siderules}
  Какие задачи решают уровни эталонной модели ВОС?
\end{siderules}

В эталонной модели ВОС существует 7 уровней, которые решают следующие задачи:
\begin{enumerate}[leftmargin=0pt]
  \item \textbf{Физический уровень}. На этом уровне происходит превращение битов
  в сигналы, которые затем передаются по среде (оптические, электрические или
  радио сигналы). Другими словами, физический уровень отвечает за обмен
  информацией с помощью физических сигналов между физическими устройствами.
  \item \textbf{Канальный уровень}. На этом уровне происходит преобразование
  потока бит в кадры, обнаружение и коррекция ошибок, решается проблема
  адресации при передаче информации. Также в случае, если один канал связи
  используется сразу несколькими устройствами, канальный уровень производит
  проверку доступности среды.
  \item \textbf{Сетевой уровень}. На этом уровне решается проблема
  маршрутизации, то есть выбора оптимального пути передачи данных. В добавок к
  физической адресации, которая появилась на канальном уровне, на сетевом уровне
  добавляется логическая адресация.
  \item \textbf{Транспортный уровень}. На этом уровне происходит установка
  соединений и управление потоком данных. Также транспортный уровень отвечает за
  надежность доставки пакетов, за то, что пакеты будут доставлены в правильном
  порядке, не будут продублированы или потеряны.
  \item \textbf{Сеансовый уровень}. На этом уровне происходит установка сеансов
  связи: фиксируется, какая из сторон является активной в настоящий момент,
  предоставляются инструменты для синхронизации данных.
  \item \textbf{Представительный уровень}. На этом уровне решается проблема
  представления данных, понятных как отправителю, так и получателю. Примерами
  представлений являются различные кодировки текста и шифрование.
  \item \textbf{Прикладной уровень}. На этом уровне работают пользовательские
  приложения и сервисы, которые используют сеть. Примерами протоколов
  прикладного уровня являются HTTP, FTP, SMTP.
\end{enumerate}

\section*{Вопрос №3}

\begin{siderules}
  В чем сущность метода коммутации пакетов?
\end{siderules}

Сущность метода коммутации пакетов заключается в следующем. Пусть необходимо
отправить по сети достаточно большой объем информации, который не помещается в
один кадр. Для того, чтобы передать эту информацию, разделим ее на порции,
называемые <<пакетами>>. Каждый такой пакет будет снабжен заголовками, в которых
содержится адрес назначения и другая информация, которая используется при
доставке пакета. Разбиение на пакеты происходит в четыре этапа:
\begin{enumerate}
  \item отправитель разделяет исходную информацию на части;
  \item к каждой части добавляются заголовки и служебная информация;
  \item каждая часть отправляется, получатель принимает каждую часть, если
  необходимо, упорядочивает части между собой;
  \item получатель проделывает все те же действия, что и отправитель, но в
  обратном порядке.
\end{enumerate}

\section*{Вопрос №4}

\begin{siderules}
  В чем сущность дейтаграммного способа доставки сообщений?
\end{siderules}

Дейтаграммный способ доставки сообщений --- это способ передачи данных в виде
отдельных и несвязанных между собой пакетов. При этом ни очередность поступления
пакетов, ни надежность их доставки не гарантируется при дейтаграммном способе.
Таким образом, в случае дейтаграммного способа пакет, отправленный первым (или
любым другим), может прийти к получателю последним или не прийти вовсе.
Благодаря тому, в дейтаграммном способе отсутствует механизм контроля
целостности данных, для него характерна высокая скорость передачи данных и
низкая нагрузка на сеть.

\end{document}
